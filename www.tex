\documentclass[11pt]{beamer}
\usepackage[utf8]{inputenc}
\usepackage[T1]{fontenc}
\usepackage{lmodern}
\usetheme{Berkeley}
\begin{document}
\title{Desarrollo Web FullStack (Backend \& Frontend). Base de Datos}
\author{Ricardo Michel Mallqui Baños}
	%\subtitle{}
	\logo{
	\includegraphics[width=1cm]{UNSCH}
}
	\institute{UNIVERSIDAD NACIONAL AMZÓNICA DE MADRE DE DIOS-FACULTAD DE INGENIERIA}
	%\date{}
	%\subject{}
	%\setbeamercovered{transparent}
	%\setbeamertemplate{navigation symbols}{}
	\begin{frame}[plain]
		\maketitle
	\end{frame}

\begin{frame}
	\tableofcontents
\end{frame}
		\section{Desarrollador Full Satck}
\begin{frame}{Backend \& Frontend}
	\textbf{Frontend}-CRUD comprende todas las acciones relacionadas con el diseño de experiencia que tendrá un visitante a una página web, mientras que el \textbf{backend}-BASE DE DATOS se refiere a la estructuración del sitio y la programación de sus funcionalidades principales.
\end{frame}

	
	\begin{frame}
		\frametitle{Desarrollador Full Satck}
Un desarrollador \textbf{Full Satck} es un profesional híbrido, con un perfil técnico muy completo, que tiene experiencia implementando soluciones en la parte del cliente \textbf{Front End} y el servidor \textbf{Back End} y dispone de capacidad para comunicarse de forma efectiva con el resto del equipo. Funciones:

\begin{enumerate}
	\item Desarrollar la arquitectura del sitio web.
	\item Diseñar interfaces de usuario y otros componentes Front End.
	\item Desarrollar aplicaciones Back End.
	\item Crear bases de datos y servidores.
	\item Supervisar la velocidad y escalabilidad del software.
	\item Gestionar proyectos complejos.
	\item Asesorar a otros departamentos.
\end{enumerate}

	\end{frame}
\section{Lenguajes y herramientas para ser Full Stack}
\begin{frame}{Lenguajes y herramientas para ser Full Stack}
	\begin{enumerate}
		\item Git
		\item Lenguajes de programación Front End
		\item Lenguajes y frameworks de Back End
		\item Bases de datos
		\item Arquitectura web
	\end{enumerate}
\end{frame}

\begin{frame}{Frameworks, librerías o preeprocesadores}
	Dominando estas tecnologías puedes usar algunos frameworks, librerías o preeprocesadores que expanden tus capacidades para crear todo tipo de interfaces de usuario. Algunos de ellos son:
	\begin{enumerate}
\item 	React
	\item Vue
	\item Angular
	\item Svelte
	\item Bootstrap
	\item Foundation
	\item Sass, Less, Stylus
	\item PostCSS
	\end{enumerate}


\end{frame}

\begin{frame}{Staks}
	Los stacks, también llamados ‘ecosistemas de datos’, son los conjuntos de servicios tecnológicos utilizados para construir y ejecutar una única aplicación. 
	\begin{enumerate}
		\item MERN Stack (Es un conjunto de subsistemas de software para el desarrollo de aplicaciones, que incluye las tecnologías \textbf{Mongo DB}, \textbf{Express.js}, \textbf{Vue.js} y \textbf{Node.js})
		\item MEVN ({Mongo DB}, {Express.js}, \textbf{React.js} y {Node.js})
		\item MEAN ({Mongo DB}, {Express.js}, \textbf{Angular.js} y {Node.js})
	\end{enumerate}
\end{frame}


\section{Base de datos}
\begin{frame}{Base de datos}
	Aquel conjunto de datos almacenados y estructurados según sus características o tipología para ser utilizados o consultados posteriormente.
	
	
	\begin{alertblock}{Objetivos de las bases de datos}
		-Trabajar con consultas complejas y no predefinidas
		-Ofrecer flexibilidad e independencia
		-Evitar la redundancia
		-Garantizar la integridad de los datos
		-Permitir la concurrencia de usuarios
		-Asegurar la seguridad de los datos
	\end{alertblock}
	
	
\end{frame}

\begin{frame}
	\begin{block}{Tipos de bases de datos}
	\begin{enumerate}
		\item Según el modelo
		-Relacionales
		-Distribuida
		-NoSQL
		-Orientadas a objetos
		-Multidimensionales
		-Documentales
		-Deductivas
		-Transaccionales
		-Jerárquicas
		-Red
		\item 		Según el contenido
		-Bibliográficas
		-Texto completo
		\item 	Según la variabilidad
		Estáticas
		Dinámicas
	\end{enumerate}
\end{block}

	\begin{block}{Principales motores de bases de datos}
		-MySQL
		-SQLite
		-MongoDB
		-MariaDB
	\end{block}
\end{frame}

\begin{frame}{Referencias}
	\begin{itemize}
		\item https://platzi.com/blog/que-es-frontend-y-backend/
		\item https://profile.es/blog/desarrollador-full-stack/
		\item https://jmsolera.com/que-es-mern/
	\end{itemize}
\end{frame}


\begin{frame}
	\huge Gracias!!!
\end{frame}

\end{document}